\section{Conclusion}\label{sec:conclusion}

The technique in this paper allows a simple
specification of a deterministic fragment of the Java library and
enforces its run-time constraints on values.
The technique is general, although we focused on the
relevant issue of avoiding \<Object.hashCode()>'s non-determinism.
For that special case, it could be possible to patch the Java library
with an implementation of \<Object.hashCode()> that constantly and
deterministically yields a constant. That would work for this run-time condition
but would be invasive (all blockchain nodes should use a patched,
unofficial version of the Java library). It would also lead to very inefficient
code if objects are put in hashsets without redefining their \<hashCode()>.
Our solution rejects the smart contracts in that case, statically or dynamically.

Our specification and verification technique does not help with the
identification of deterministic fragments of the Java library, that remains
an open problem. Currently,
we have checked the source code of the library to convince ourselves that,
for instance, the fragment in Figs.~\ref{fig:white_listed_Object}, \ref{fig:white_listed_Collection},
\ref{fig:white_listed_Set}, \ref{fig:white_listed_List},
\ref{fig:white_listed_HashSet} and~\ref{fig:white_listed_ArrayList} is deterministic.
Static analysis could help here, if it could prove some fragment deterministic,
automatically. But the complexity of the library code, the need to infer
sensible run-time conditions and the use of native code seem to make this task very hard.
