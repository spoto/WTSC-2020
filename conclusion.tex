\section{Conclusion}\label{sec:conclusion}

The technique in this paper allows a simple
specification of a deterministic fragment of the Java library and
enforces its run-time constraints.
Experiments show that it works in practice and does not incur
in size or time degradation of the compiled code.
Current work consists in enlarging the white-listed fragment.
We will perform this task on demand, while writing smart contracts
in Takamaka, in order to concentrate only on library portions
that are relevant for that.

%Our specification and verification technique does not help with the
%identification of deterministic fragments of the Java library, that remains
%an open problem. Currently,
%we have checked the source code of the library to convince ourselves that,
%for instance, the fragment in Figs.~\ref{fig:white_listed_Object}, \ref{fig:white_listed_Object_Collection_Set}
%and~\ref{fig:white_listed_List_HashSet_ArrayList} is deterministic.
%Static analysis could help here, if it could prove some fragment deterministic,
%automatically. But the complexity of the library code, the need to infer
%sensible run-time conditions and the use of native code seem to make this task very hard.
