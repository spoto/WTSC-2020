\section{Determinism for Java Methods in Smart Contracts}\label{sec:determinism}

This section discusses the different kinds of determinism of methods from
the Java library and what must be required if they are used
in a language for smart contracts.
We recall that the Java library exists in different versions, from its first
$1.0$ edition of 1996 to its $13$ edition of September 2019.
Each version has different implementations, with OpenJDK probably being
the most used, nowadays. Some classes and methods exist only in some versions of the library.
For instance, class \<java.lang.Integer> exists from the very first $1.0$ edition,
while class \<java.util.Collection> was only introduced in version $1.2$.
In the following, when discussing a specific method, we assume that a version
of the Java library is used, where the method (and hence its declaring class) exists.
We report fully-qualified names of classes only the first time they are cited.

Let us discuss the same meaning of \emph{determinism}.
The \emph{state} of the Java interpreter is typically defined
as a function from variables and object fields into values.
Then the goal is to guarantee that the execution $\phi$ of a piece of code,
from the same state $\sigma$ but in two different blockchain nodes, yields the same result, \emph{i.e.},
same next state $\sigma'$ and same value $r$ (if the code is an expression):
$\phi(\sigma)=\langle\sigma',r\rangle$ in all nodes of the network.
This definition, however, is too strict. Consider the constructor method
\<String()> of \<java.lang.String>, that instantiates and returns a new empty \<String>
object. It is an expression whose execution,
from any given state $\sigma$ but in two different blockchain nodes,
will very likely yield two distinct
pointers $r$ in RAM: the heap allocation will likely
select two distinct free locations for the new object.
But both references will refer to a brand new empty string.
As long as the programming language does not allow one to distinguish
the exact pointers, but only the object it refers to,
then such two next states and results can be considered \emph{equivalent}.
This notion of state and reference equivalence is not new, but actually borrowed
from~\cite{BanerjeeN05,BarthePR13}.
With this interpretation in mind, method \<String()>, applied to two equivalent
states but in two distinct blockchain nodes,
yields equivalent states and equivalent results (two references to
brand new empty strings), and is consequently deterministic.

\begin{figure}[t]
  \begin{center}
    \begin{tabular}{|l||C{1cm}|C{1.25cm}|C{1.57cm}|C{1cm}|}
      \hline
      \multicolumn{1}{|c||}{\multirow{3}{*}{\textbf{Method}}} & \multicolumn{4}{ c| }{\textbf{Deterministic?}}\\\cline{2-5}
      & \multirow{2}{*}{always} & \multirow{2}{*}{platform} & platform + & \multirow{2}{*}{never} \\
      & & & conditions & \\\hline\hline
      \<Number.intValue()> & \checkmark &   &   & \\\hline
      \<Object()> & \checkmark &   &   & \\\hline
      \<String()> & \checkmark &   &   & \\\hline
      \<HashSet$\text{<}$E$\text{>}$()> & \checkmark &   &   & \\\hline
      \<String(String original)> & \checkmark &   &   & \\\hline
      \<String.concat(String other)> & \checkmark &   &   & \\\hline
      \<String.length()> & \checkmark &   &   & \\\hline
      \<Integer.valueOf(int i)> & & \checkmark & & \\\hline
      \<Object.hashCode()> & & & \checkmark & \\\hline
      \<Object.toString()> & & & \checkmark & \\\hline
      \<Collection$\text{<}$E$\text{>}$.iterator()> & & & \checkmark & \\\hline
      \<Collection$\text{<}$E$\text{>}$.stream()> & & & \checkmark & \\\hline
      \<Collection$\text{<}$E$\text{>}$.add(E e)> & & & \checkmark & \\\hline
      \<StreamSupport.stream(Spliterator$\text{<}$T$\text{>}$ s, boolean p)> & & & \checkmark & \\\hline
      \<System.currentTimeMillis()> & & & & \checkmark \\\hline
      \<BaseStream$\text{<}$T,S$\text{>}$.parallel()> & & & & \checkmark \\\hline
      \<Thread.start()> & & & & \checkmark \\\hline
      \multicolumn{1}{c}{} & \multicolumn{3}{c}{\upbracefill}\\
      \multicolumn{1}{c}{} & \multicolumn{3}{c}{potentially white-listed}\\
    \end{tabular}
  \end{center}
  \caption{Some methods of the Java library, with their behavior \emph{wrt.} determinism.
\emph{Always} means that a method does not compromise code determinism.
\emph{Platform} means that
a method does not compromise determinism, but only if a specific implementation
of the library is fixed. \emph{Platform + conditions} means that a method
does not compromise determinism, but only if an implementation of the library is fixed
\emph{and} the program satisfies extra conditions, that typically refer to
the actual arguments passed to that method or to other methods at run time.
\emph{Never} means
that a method can have different behaviors in distinct executions,
even on a specific library implementation, and no
condition can be sensibly devised to make its behavior deterministic.}\label{fig:determinism}
\end{figure}

Fig.~\ref{fig:determinism} reports some examples of methods
from the Java library, classified
on the basis of the kind of determinism that holds for them.
We discuss them below. We recall that, in Java, a method call
\<o.m(pars)> specifies its \emph{receiver} \<o> and
its \emph{static target} \<C.m(types)>, that is, a method signature.
Since Java is an object-oriented language,
a non-static method call runs the implementation of \<C.m(types)>
that is selected by looking up \<m(types)> from the run-time class of \<o> upwards,
along the superclass chain.\footnote{Java also allows so-called \emph{special} calls, such as
   \<super.m()>, that start the implementation look-up from a given \emph{static}
   type; as well as calls embedded in closures, such as method references
   (corresponding to \<invokedynamic> in Java bytecode). For simplicity,
   these calls are not discussed here, but our implementation deals with them.}
Hence, any implementation of \<m(types)> in \<C> or in any of its
subtypes can be run. For instance, a call \<o.intValue()> with static target \<Number.intValue()>
can be called on any object \<o> that extends
\<java.lang.Number>. At run time, it might
execute any implementation of \<intValue()> in any subtype of \<Number>,
such as in \<Integer> or in \<java.lang.Double>, depending on the run-time type of \<o>.
Hence, when Fig.~\ref{fig:determinism} classifies \<Number.intValue()> as always
deterministic, this applies to every implementation of
\<intValue()> in the subtypes of \<Number> of the Java library.

\subsubsection*{Always Deterministic Methods.}
Many methods of the Java library are clearly deterministic. Their behavior is fixed
by the official documentation by Oracle
and does not change across distinct versions of the library. An example is the unwrapping
method \<intValue()> of \<Number> (Fig.~\ref{fig:determinism}). It yields
the primitive \<int> value corresponding to an instance of the abstract class \<Number>, such
as objects of class \<Integer> or \<Double>. For instance,
\[
\<new Integer(3).intValue() == 3>
\]
holds in any version of the Java library, as well as
\[
\<new Double(3.14).intValue() == 3>
\]
since, in Java, truncation of \<double> into \<int> is machine-independent.
Hence, a language for smart contracts, that must require determinism,
can safely allow the use of that method, always. Other examples
are the constructors of \<String>, \<java.lang.Object> and \<java.util.HashSet>
reported in Fig.~\ref{fig:determinism}, or methods
\<String.concat(String other)>, that yields the concatenation of strings \<this> and \<other>,
and \<String.length()>, that yields the length of a string.
If only these methods are used, then any Java library can be used by any node of a blockchain,
even distint versions in distinct nodes.

\subsubsection*{Platform-Deterministic Methods.}
The static method \<Integer.valueOf(int i)> wraps
a primitive \<int> value into an object of class \<Integer>. It might be surprising,
but its result can be different in two blockchain nodes, if they
use distinct implementations of the Java library. For instance, while
\[
\<Integer.valueOf(3) == Integer.valueOf(3)>
\]
holds in every implementation of the library, since the official documentation requires this method
to cache values between $-128$ and $127$, inclusive, there is instead no guarantee that caching is
used outside that range. Hence
\[
\<Integer.valueOf(2019) == Integer.valueOf(2019)>
\]
might be true in some implementations of the Java library
and false in others\footnote{For instance, it is false in Oracle JDK 13.0.1.}, that yield distinct objects for the two calls.
We call \emph{platform deterministic} such methods, since they are deterministic only once a specific
implementation (\emph{platform}) of the Java library is fixed.
If only methods of this and of the previous categories are used, then
all nodes of a blockchain must be run on a given, fixed Java library, to guarantee
determinism. This could be part of the consensus rules.

\subsubsection*{Platform-Deterministic Methods under Specific Conditions.}
Consider method \<Object.hashCode()> of class \<Object> now. It yields an \<int>
hash of its receiver. Its implementations computes that hash from the
RAM pointer value of the object reference.
Since two blockchain nodes will likely use different RAM pointers,
this method is non-deterministic.
In other terms, this method exposes an execution detail (the exact RAM pointer)
that was meant to be invisible thanks to the notion of state equivalence used
to define determinism.
For instance, the following code, in two distinct blockchain nodes
(and even in repeated executions on the same node),
will likely assign different values to variable \<i>:
%
\begin{verbatim}
int i = new Object().hashCode();
\end{verbatim}
%
The same problem occurs for method \<Object.toString()> that, inside
\<Object>, is implemented in terms of \<Object.hashCode()> (its
implementation concatenates
the name of the run-time class of its receiver with its hash and returns it).
Hence, the following code, in two distinct blockchain nodes,
will likely assign different strings to variable \<s>:
%
\begin{verbatim}
String s = new Object().toString();
\end{verbatim}
%
However, banning calls to \<Object.hashCode()> or \<Object.toString()>
from smart contracts would be unacceptable to programmers, that
heavily use such calls in their programs, without incurring in
non-determinism. The reason is that programmers normally
take care of calling such methods only on objects that redefine
the default (non-deterministic) implementation of
\<hashCode()> and \<toString()> in class \<Object>.
If that is the case, the calls will actually
execute the redefinitions, that are typically deterministic. Hence, it seems sensible
to allow calls to such methods in smart contracts, but only if their receiver
redefines them in a deterministic way, as in the following example:

{\small
\begin{verbatim}
String o = ... // deterministic computation
int h = o.hashCode(); // o holds a String, that redefines hashCode()
\end{verbatim}}

\noindent
or even in the following:

{\small
\begin{verbatim}
String s = ... // deterministic computation
Object o = s;
int h = o.hashCode(); // o holds a String, that redefines hashCode()
\end{verbatim}}

\noindent
In both cases, variable \<o>, at run time,
holds a string and the deterministic (but potentially platform-specific)
redefinition of \<hashCode()> inside \<String> will be called.
Hence, such methods are \emph{platform-deterministic under certain conditions}:
they are deterministic if a given Java library is fixed \emph{and} some run-time conditions hold.
Sec.~\ref{sec:enforcing} shows how such conditions can be enforced.

Consider methods \<iterator()> and \<stream()> of the generic class
\<Collection$\text{<}$E$\text{>}$> now. They provide two different ways for
processing the elements, of type $E$, of a collection. The former implements
the traditional \emph{iterator pattern} and yields an object able to enumerate
the elements. The latter yields a \emph{stream}, that is,
an abstract description of a lazy algorithm on the
elements, that can be subsequently programmed and executed.
Streams implement the map/filter/reduce design pattern. Their specification
makes heavy use of lamba expressions.
The interesting point is that neither method
guarantees any specific enumeration order: the elements will be
enumerated in any order, in general. There are collections for which they
guarantee an order, such as instances of \<java.util.List$\text{<}$E$\text{>}$>:
on lists, enumeration proceeds from head to tail;
or instances of \<java.util.LinkedHashSet$\text{<}$E$\text{>}$>, on which they proceed in
insertion order.
For collections such as \<java.util.HashSet$\text{<}$E$\text{>}$>, instead, the order
varies with the library version \emph{and} with each execution. The reason is that
\<HashSet> uses a hashmap~\cite{CormenLRS09}
to store elements with the same \<hashCode()> inside the same bucket.
Since \<hashCode()>, as shown above, is in general non-deterministic, then
the distribution of elements inside the buckets varies from run to run and their enumeration
as well, being implemented as a lexicographic scan of the buckets' elements.
One could forbid \<HashSet> and only allow its
more expensive sibling \<LinkedHashSet>, whose iteration order is fixed.
But then method \<add()> would still be non-deterministic, since its gas consumption
is affected by the shape of the buckets, as discussed later.
Moreover, programmers like \<HashSet>, that is a very frequently class used
in Java code, and they would be annoyed if it would be
forbidden. It is much better to observe that,
if the \<hashCode()> of all its elements has been redefined in a deterministic way
and if a specific library version is fixed, then the behavior of \<HashSet> becomes
deterministic, since the shape of the buckets is the same for every run. This means that one
can allow, in smart contracts, calls to \<iterator()> and \<stream()> on any collection,
but only under such conditions. Sec.~\ref{sec:enforcing} shows how they can be enforced.

Consider method \<add(E e)> of \<Collection$\text{<}$E$\text{>}$>.
It adds an element \<e> to the collection. For a given library version,
it is deterministic on lists: it adds \<e> to the end of the list.
However, on \<HashSet> and \<LinkedHashSet> it scans the bucket selected
for \<e.hashCode()> and checks if an equal element was already in that
bucket, by calling \<equals()> against each of its elements. Hence \<add(E e)>
on a hashset consumes an amount of gas that depends on the shape of its buckets.
Again, the solution is to require that all elements of the set and \<e> redefine
\<hashCode()> in a deterministic way, which must be enforced at run time
(Sec.~\ref{sec:enforcing}).

Consider the static method
\<stream(Spliterator$\text{<}$T$\text{>}$ s, boolean p)>
of class \<java.util.stream.StreamSupport>. It
yields a stream for processing the elements specified by
the given \<Spliterator>. If the \<Spliterator> is deterministic,
the resulting stream is deterministic as well,
on a given library version, but only if it is sequential.
Passing true for \<p> would yield a parallel stream instead, that is
inherently non-deterministic and cannot be allowed in smart contracts.
Hence, this method can be used but only if a specific library is used and
if it is enforced that false is passed for \<p> at run time (Sec.~\ref{sec:enforcing}).

If only methods of this and of the previous two categories are used, then
all nodes of a blockchain must be run on a given, fixed Java library, \emph{and}
the run-time conditions that entail determinism must be somehow enforced.

\subsubsection*{Never Deterministic Methods.}
Static method \<System.currentTimeMillis()>, in \<java.lang.System>,
yields the number of milliseconds elapsed since the beginning of 1970.
Not surprisingly, code using that method will likely get different measures in different
blockchain nodes.
Such an inherently non-deterministic method cannot be used in a smart contract.

Consider method \<parallel()> of \<java.util.stream.BaseStream> now.
It yields a parallel version of a stream. For instance, consider the following code:
%
\begin{verbatim}
List<Integer> list = ...
int positive = list.stream().parallel()
  .mapToInt(Integer::intValue)
  .filter(i -> i > 0)
  .findAny()
  .getAsInt();
\end{verbatim}
%
and assume that \<list> contains at least two distinct positive \<Integer>s.
This code processes \<list> with a parallel algorithm
that unwraps each \<Integer> element of the list into its corresponding \<int> primitive value,
filters only the positive values and selects any of them.
The result of \<findAny()> is an optional value, hence the
\<getAsInt()> call at the end. Since the algorithm is
parallel, each execution of this code might select a different
positive element, depending on thread scheduling. Hence, \<parallel()>
introduces non-determinism and
cannot be allowed in smart contracts. Note that even the gas consumption
of the code is not deterministic, since it depends on how many elements
are checked before a thread encounters a positive value and terminates
the look-up.

In general, methods that introduce parallelism are never deterministic.
Another example is \<Thread.start()> in \<java.lang.Thread>, that spawns
a parallel execution thread.

If methods of this category are used, then the code cannot be used in blockchain,
since there is not way to make it deterministic.
