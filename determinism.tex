\section{Determinism for Java Methods in Smart Contracts}\label{sec:determinism}

This section discusses the different kinds of determinism of methods from
the Java library and what must be required if such methods are used
in a language for smart contracts.
We recall that the Java library exists in different versions, from its first
$1.0$ edition of 1996 to its $13$ edition expected to ship in September 2019.
Each version has different implementations, with OpenJDK probably being
the most used, nowadays. Some classes and methods exist only in some versions of the library.
For instance, class \<java.lang.Integer> exists from the very first $1.0$ edition,
while class \<java.util.Collection> was only introduced in version $1.2$.
In the following, when discussing a specific method, we assume that a version
of the Java library is used, where the method (and hence its declaring class) exist.
Moreover, we report fully qualified names of classes only the first time they are cited.

Let us discuss the same meaning of \emph{determinism}. Remember that
our goal is to guarantee that the code of a given smart contract, executed
in two different blockchain nodes, yields the same result.
If we define the state
of the Java interpreter as a function from variables and object fields into values, then
determinism means that a method, applied at a given state, can only yield a single result,
that is, a single pair of subsequent state and return value (if any).
This definition, however, is too strict. Consider the constructor method
\<String()> of \<java.lang.String>, that constructs and returns a new empty \<String>
object. The execution of this method, in two different blockchain nodes,
will very likely yield distinct references (\emph{i.e.}, pointers in RAM).
But both references will refer to a brand new empty string.
As long as the programming language does not allow one to see
the exact pointer value of a reference, but only the object it refers to,
then two states can be considered equivalent if and only if there is
a bidirectional mapping between their references.
With this interpretation of \emph{determinism}, method \<String()>, applied to two equivalent
states, yields equivalent states and equivalent return value (a reference to
a brand new empty string).
Hence, that constructor method is deterministic.

\begin{figure}[t]
  \begin{center}
    \begin{tabular}{|l||C{1cm}|C{1.25cm}|C{1.57cm}|C{1cm}|}
      \hline
      \multicolumn{1}{|c||}{\multirow{3}{*}{\textbf{Method}}} & \multicolumn{4}{ c| }{\textbf{Deterministic?}}\\\cline{2-5}
      & \multirow{2}{*}{always} & \multirow{2}{*}{platform} & platform + & \multirow{2}{*}{never} \\
      & & & conditions & \\\hline\hline
      \<Number.intValue()> & \checkmark &   &   & \\\hline
      \<Object()> & \checkmark &   &   & \\\hline
      \<String()> & \checkmark &   &   & \\\hline
      \<HashSet$\text{<}$E$\text{>}$()> & \checkmark &   &   & \\\hline
      \<String(String original)> & \checkmark &   &   & \\\hline
      \<String.concat(String other)> & \checkmark &   &   & \\\hline
      \<String.length()> & \checkmark &   &   & \\\hline
      \<Integer.valueOf(int i)> & & \checkmark & & \\\hline
      \<Object.hashCode()> & & & \checkmark & \\\hline
      \<Object.toString()> & & & \checkmark & \\\hline
      \<Collection$\text{<}$E$\text{>}$.iterator()> & & & \checkmark & \\\hline
      \<Collection$\text{<}$E$\text{>}$.stream()> & & & \checkmark & \\\hline
      \<Collection$\text{<}$E$\text{>}$.add(E e)> & & & \checkmark & \\\hline
      \<StreamSupport.stream(Spliterator$\text{<}$T$\text{>}$ s, boolean p)> & & & \checkmark & \\\hline
      \<System.currentTimeMillis()> & & & & \checkmark \\\hline
      \<BaseStream$\text{<}$T,S$\text{>}$.parallel()> & & & & \checkmark \\\hline
      \<Thread.start()> & & & & \checkmark \\\hline
      \multicolumn{1}{c}{} & \multicolumn{3}{c}{\upbracefill}\\
      \multicolumn{1}{c}{} & \multicolumn{3}{c}{potentially white-listed}\\
    \end{tabular}
  \end{center}
  \caption{Some methods of the Java library, with their behavior \emph{wrt.} code determinism.
\emph{Always} means that a method does not compromise code determinism.
\emph{Platform} means that
a method does not compromise determinism, but only once a specific implementation
of the library is fixed. \emph{Platform + conditions} means that a method
does not compromise determinism, but only if an implementation of the library is fixed
\emph{and} the program satisfies extra conditions, that typically refer to
the actual arguments passed to the method at run time or to if and how
other methods are called. \emph{Never} means
that a method can have different behaviors in distinct executions,
even after fixing the library implementation of the actual arguments, and no
condition can be sensibly devised to guarantee that it does not compromise code
determinism.}\label{fig:determinism}
\end{figure}

Fig.~\ref{fig:determinism} reports some examples of methods
from the Java library, classified
on the basis of the kind of determinism that holds for them.
We discuss them below. Since Java is an object-oriented language,
a non-static method can always be called on any object whose class
is a subtype of that where the method is defined. For instance, method \<Number.intValue()> in
Fig.~\ref{fig:determinism} can be called on any object \<o> that extends
\<java.lang.Number>, as in \<o.intValue()>. That call, at run time, might
execute any implementation of \<intValue()> in any subtype of \<Number>
(such as in \<java.lang.Integer> or in \<java.lang.Double>), depending on the run-time
type of the value in \<o>. That value, that determines the implementation of the call, is called
\emph{receiver} of the call.
The implementation is selected by starting from the run-time type of the receiver
and looking-up for the implementation along the superclass
chain.\footnote{Java also allows the so-called \emph{special} calls, such as
   \<super.m()>, that start the look-up from a given \emph{static}
type, as well as calls embedded in closures, such as method references
(corresponding to \<invokedynamic> in Java bytecode). For simplicity,
these calls are not discussed here, but our implementation deals with them.}
Hence, when Fig.~\ref{fig:determinism} classifies \<Number.intValue()> as always
deterministic, this applies to every implementation of
\<intValue()> in the subtypes of \<Number> of the Java library.

\subsubsection*{Always Deterministic Methods.}
Many methods of the Java library are clearly deterministic. Their behavior is fixed
and does not change with distinct versions of the library. An example is the unwrapping
method \<intValue()> of \<Number> (Fig.~\ref{fig:determinism}). It yields
the primitive \<int> value corresponding to an instance of the abstract class \<Number>, such
as objects of class \<java.lang.Integer> or \<java.lang.Double>. For instance,
\[
\<new Integer(3).intValue() == 3>
\]
holds in any version of the Java library, as well as
\[
\<new Double(3.14).intValue() == 3>
\]
since truncation of \<double> into \<int> is machine-independent in Java.
Hence, a language for smart contracts, that must require determinism,
can safely allow the use of that method, always. Other examples of such methods
are the constructors of \<String>, \<java.lang.Object> or \<java.util.HashSet>
reported in Fig.~\ref{fig:determinism}, or methods
\<String.concat(String other)>, that yields the concatenation of \<this> and \<other>,
and \<String.length()>, that yields the length of a string.

\subsubsection*{Platform-Deterministic Methods.}
The static method \<Integer.valueOf(int i)> performs the inverse operation of wrapping
a primitive \<int> value into an object of class \<Integer>. It might be surprising,
but its result can be different in two blockchain nodes, if they
use distinct implementations of the Java library. For instance,
\[
\<Integer.valueOf(3) == Integer.valueOf(3)>
\]
holds, in every implementation of the library, since the official documentation requires this method
to cache values between $-128$ and $127$, inclusive. But there is no guarantee that caching is
used outside that range, so that
\[
\<Integer.valueOf(2019) == Integer.valueOf(2019)>
\]
might be true in some implementations of the Java library
and false in others, that yield distinct objects for the two calls.
We call \emph{platform deterministic} such methods, since they are deterministic only once a specific
implementation (\emph{platform}) of the Java library is fixed. Such methods can be used in smart contracts,
but only if all blockchain nodes are forced to
use a specific version of the Java library, which can be part of the consensus rules.

\subsubsection*{Platform-Deterministic Methods under Specific Conditions.}
Consider method \<Object.hashCode()> of \<Object> now. It yields an \<int>
hash of an object. In its current implementation, it computes the hash from the
RAM pointer value of the object reference.
Since two blockchain nodes will likely get different RAM pointers,
this method is non-deterministic.
In other terms, this methods exposes an execution detail (the exact RAM pointer)
that was meant to be abstracted away in our definition of determinism.
For instance, the following code, in two distinct blockchain nodes,
and even on different executions on the same node,
will likely assign different values to variable \<i>:
%
\begin{verbatim}
int i = new Object().hashCode();
\end{verbatim}
%
The same problem occurs for method \<Object.toString()> that, inside
\<Object>, is implemented in terms of \<Object.hashCode()> (in OpenJDK 11, its
implementation concatenates
the name of the run-time class of the object with its hash and returns it).
Hence, the following code, in two distinct blockchain nodes,
will likely assign different strings to variable \<s>:
%
\begin{verbatim}
String s = new Object().toString();
\end{verbatim}
%
However, banning calls to \<Object.hashCode()> or \<Object.toString()>
from smart contracts would be unacceptable to programmers, that
heavily use such calls in their programs, without incurring in
non-determinism. The reason is that programmers normally
take care of calling such methods only on objects that redefine
the default (non-deterministic) implementation of
\<hashCode()> and \<toString()> from class \<Object>.
If that is the case, the call will actually
execute the redefinition, that is typically deterministic. Hence, it seems sensible
to allow calls to such methods in smart contracts, but only if they occur
on objects that redefine them in a deterministic way, as in the following example:
%
\begin{verbatim}
String o = ... // deterministic computation
int h = o.hashCode();
\end{verbatim}
%
or even the following:
%
\begin{verbatim}
String s = ... // deterministic computation
Object o = s;
int h = o.hashCode();
\end{verbatim}
%
In both cases, variable \<o>, at run time,
refers to a string and the deterministic (but potentially platform-specific)
redefinition of \<hashCode()> inside \<String> will be called.
This is why we call such methods \emph{platform-deterministic under certain conditions}:
they are deterministic if a given Java library is fixed \emph{and} some run-time conditions hold.
Sec.~\ref{sec:enforcing} shows how this extra condition on the run-time class of
\<o> can be enforced.

Consider methods \<iterator()> and \<stream()> of the generic class
\<Collection$\text{<}$E$\text{>}$> now. They provide two different ways for
processing the elements, of type $E$, of a collection. The former implements
the traditional \emph{iterator pattern} and yields an object able to enumerate
the elements. The second yields an abstract description of an algorithm on the
elements, that can be subsequently programmed and executed. The interesting point, is
that they do not guarantee any specific enumeration order: the elements can be
provided in any order, in general. There are, however, some collections for which they
guarantee an order, such as instances of \<java.util.List$\text{<}$E$\text{>}$>:
on lists, they are guaranteed to proceed from the first towards the last element of the list;
or instances of \<java.util.LinkedHashSet$\text{<}$E$\text{>}$>, on which they proceed in
insertion order.
For other collections, though, such as for \<java.util.HashSet$\text{<}$E$\text{>}$>, the order
varies with the version of the library \emph{and} with each execution. The reason is that
\<HashSet> uses a hashmap to store elements with the same \<hashCode()> inside the same bucket.
Since the \<hashCode()> method, in general, as shown above, is non-deterministic, it follows
that the distribution of elements inside the buckets varies from run to run and their enumeration
as well, being implemented as a lexicographic enumeration of the buckets' elements.
One could decide to forbid class \<HashSet> and only allow its
more expensive sibling \<LinkedHashSet>. But that would not be enough, as we will discuss later
for the \<add()> method. Moreover, programmers like \<HashSet> and would be annoyed if it would be
forbidden. It is much better to observe that, if the \<hashCode()> of all its elements has been
redefined in a determinitsic way
and a specific library version is fixed, then the behavior of \<HashSet> becomes
deterministic, since the shape of the buckets is the same for every run. This means that one
can allow \<iterator()> and \<stream()> calls on collections
in smart contracts if a library version is fixed and if
it is possible to guarantee that all elements of the collection redefine \<hashCode()>
in a deterministic way. Sec.~\ref{sec:enforcing} shows how this extra run-time condition
can be enforced.

Consider method \<add(E e)> of \<Collection$\text{<}$E$\text{>}$> now.
It adds an element \<e> to the collection. Fixed a given library version,
its behavior is deterministic on lists: it adds \<e> to the end of the list.
However, on \<HashSet> and \<LinkedHashSet> it scans the bucket selected
for the \<e.hashCode()> and checks if an equal element was already in the
bucket, by calling \<equals()> against each element in the bucket. Hence \<add(E e)>
on a hashset consumes an amount of gas that depends on the shape of its buckets.
Again, the solution is to require that all elements of the set and \<e> redefine
\<hashCode()> in a deterministic way, which must be enforced at run time
(Sec.~\ref{sec:enforcing}).

Consider the static method
\<stream(Spliterator$\text{<}$T$\text{>}$ s, boolean p)>
of class \<java.util.stream.StreamSupport>. It
yields a stream for processing the elements specified by
the given \<Spliterator>. If the \<Spliterator> is deterministic,
the resulting stream is deterministic as well,
on a given library version, but only if it is sequential (\<p> must be false).
Passing true for \<p> would, instead, yield a parallel stream, that is
inherently non-deterministic and cannot be allowed in smart contracts.
Hence, this method can be used but only if a specific library is used and
if it is enforced that false is passed for \<p> at run time (Sec.~\ref{sec:enforcing}).

\subsubsection*{Never Deterministic Methods.}
Static method \<System.currentTimeMillis()>, in \<java.lang.System>,
yields the number of milliseconds elapsed since the beginning of 1970.
Not surprisingly, code using that method will likely get different measures in different
blockchain nodes.
Such an inherently non-deterministic method cannot be used in a smart contract.

Consider method \<parallel()> of \<java.util.stream.BaseStream> now.
It yields a parallel version of a \emph{stream}. Introduced in Java 8, streams are
an abstract description of a lazy enumeration algorithm over values.
They are meant to implement the map/filter/reduce design pattern. Their specification
makes heavy use of lamba expressions. For instance, consider following code:
%
\begin{verbatim}
List<Integer> list = ...
int positive = list.stream().parallel()
  .mapToInt(Integer::intValue)
  .filter(i -> i > 0)
  .findAny()
  .getAsInt();
\end{verbatim}
%
and assume that \<list> contains at least a positive \<Integer>.
This code processes \<list> with a parallel algorithm
that unwraps each \<Integer> element of the list into its corresponding \<int> primitive value,
filters only the positive values and then selects any of them.
The result of \<findAny()> is an optional value, hence the
\<getAsInt()> call is needed at the end. Since the algorithm is
parallel, each execution of the code might select a different
positive element, depending on thread scheduling. Hence, \<parallel()>
introduces non-determinism and
cannot be allowed in a smart contract. Note that the same gas consumption
of the code is not deterministic, since it depends on how many elements
are checked before a thread is executed on a positive value, which terminates
the look-up.

In general, methods that introduce parallelism are never deterministic.
Another example is \<Thread.start()> in \<java.lang.Thread>, that spawns
a parallel execution thread.
