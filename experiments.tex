\section{Experiments}\label{sec:experiments}

This section reports experiments with the addition of white-listing annotations
to a set of Takamaka smart contracts. Experiments have been performed on
an Intel 4-Core i3-4150 at $3.50$GHz with 16GB of RAM, running Linux Ubuntu 18.04.
A companion archive, provided with this paper\footnote{We have submitted the archive
with this paper and expect that it has been made available to the anonymous reviewers.},
contains (1) the fragment of white-listed library methods
used in these experiments; (2) the source code of the contracts
contained in the twelve jars reported in Fig.~\ref{fig:experiments};
(3) the same jars, instrumented without the addition of the white-listing
checks; and (4) the same jars, instrumented with the addition of the white-listing checks.

\begin{figure}[t]
  \begin{center}
\begin{tabular}{|l||r|r|r||r|r||r|}
  \hline
  archive of smart contracts & \#m & \#c & \#d & without & with & instr.\ time \\\hline\hline
  \<auctions.jar> & 20 & 0 & 0 & 11997 & 11997 & 4 \\\hline
  \<basicdependencies.jar> & 38 & 3 & 0 & 7265 & 7505 & 26 \\\hline
  \<basic.jar> & 38 & 8 & 0 & 8109 & 8464 & 31 \\\hline
  \<collections.jar> & 0 & 0 & 0 & 8909 & 8909 & 4 \\\hline
  \<crowdfunding.jar> & 0 & 0 & 0 & 10043 & 10043 & 4 \\\hline
  \<io-takamaka-code.jar> & 348 & 18 & 5 & 88101 & 89177 & 223 \\\hline
  \<javacollections.jar> & 38 & 17 & 0 & 3040 & 3635 & 18 \\\hline
  \<lambdas.jar> & 45 & 3 & 0 & 4570 & 4745 & 27 \\\hline
  \<ponzi.jar> & 69 & 7 & 0 & 12964 & 13556 & 43 \\\hline
  \<purchase.jar> & 0 & 0 & 0 & 5362 & 5362 & 4 \\\hline
  \<tictactoe.jar> & 40 & 0 & 0 & 6745 & 6745 & 27 \\\hline
  \<voting.jar> & 8 & 1 & 0 & 7210 & 7305 & 20 \\\hline
\end{tabular}
  \end{center}
  \caption{The jars of smart contracts used for the experiments. Each jar
    contains one or more contracts. For each jar, \#m is the number
    of calls to constructors or methods of the Java library;
    \#c is the number of white-listing conditions that must be checked
    for those calls; \#d is the subset of \#c that has been discharged
    statically, at instrumentation time; \emph{without} is the size
    (in bytes) of the instrumented jar without the addition of run-time checks
    for the remaining $\#c - \#d$ conditions; \emph{with}
    is the size (in bytes) of the instrumented jar with the addition of the
    run-time checks for the remaining $\#c - \#d$ conditions;
    \emph{instr.\ time} is the time for computing such instrumented jars (in milliseconds).}
  \label{fig:experiments}
  \end{figure}

Fig.~\ref{fig:experiments} reports the jar archives used for the experiments.
They contain smart contracts for auctions (\<auctions.jar>), for the definition
of objects in blockchain (\<basicdependencies.jar>), for the use of such objects
as a library (\<basic.jar>), a library of collection classes explicitly developed
for Takamaka (\<collections.jar>), contracts for crowdfunding (\<crowdfunding.jar>),
the run-time support library specific of Takamaka (\<io-takamaka-code.jar>),
contracts using the Java collections (\<javacollections.jar>), contracts
testing the use of lambda expressions (\<lambdas.jar>), Ponzi contracts
(\<ponzi.jar>), a remote purchase contract (\<purchase.jar>), a tic-tac-toe
game contract (\<tictactoe.jar>) and contracts for electronic voting (\<voting.jar>).

In Fig.~\ref{fig:experiments}, column \#m reports how many calls to constructors
or methods of the Java library are used in the examples. Our implementation
determines that they are all white-listed,
but a few of them require to enforce run-time conditions for being deterministic
(Sec.~\ref{sec:enforcing}); namely, column \#c reports how many such checks are needed;
of these, our implementation could statically discharge \#d, at instrumentation time; these needn't
be checked at run time (Sec.~\ref{subsec:static_vs_dynamic}). The remaining
$\#c - \#d$ induce the addition of run-time checks in the instrumented code. Consequently,
when $\#c - \#d > 0$, extra code is instrumented into the jars, as it can be seen by comparing
columns \emph{without} and \emph{with} in Fig.~\ref{fig:experiments}. These two columns
show that the extra checks account for a very small increase of the size of the instrumented jars.
Finally, Fig.~\ref{fig:experiments} reports the time for instrumentation, in milliseconds.
We tried to compute this time for columns \emph{with} and \emph{without}, but the results
(reported in the future) were identical. This means that
checking if method calls are white-listed and possibly adding run-time checks for white-listing
conditions takes a time infinitesimal in comparison to that of the overall instrumentation
performed by Takamaka.

We have tried to determine if the extra run-time conditions for the $\#c - \#d$ run-time checks
make the code slower. For that, we ran 113 JUnit tests on the instrumented code of the jars in Fig.~\ref{fig:experiments}
(the tests are included in the companion archive). However, the total time for running the
tests (115 seconds) was the same with or without the addition of the run-time checks, which shows that
such checks do not actually slow down the execution of the code.
