\section{Introduction}\label{sec:introduction}

Smart contracts are programs that specify the effects of running blockchain transactions.
They are written in specialized languages, that take into account
the fact that they operate on data kept in blockchain. They must
support some special concepts, such as the access to the caller of a transaction,
monetary transfers between contracts
and payment of code execution through \emph{gas}.
Such concepts are not natively available
in traditional programming languages.
Instead, the Bitcoin bytecode~\cite{Antonopoulos17,Nakamoto08}
can be seen as a language for smart contracts,
although non-Turing equivalent and mostly limited to coin transfers. The more
powerful Solidity~\cite{AntonopoulosW18},
compiled into Ethereum bytecode, allows one to code
complex smart contracts, in an imperative high-level language, and is
one of the main reasons behind the success of Ethereum.

Recently, there have been efforts towards the use of traditional
high-level languages for writing smart contracts.
A notable example is Hyperledger Fabric~\cite{AndroulakiBBCCC18,Vukolic17}, that allows one
to write smart contracts in Java, among other languages. Quoting from~\cite{AndroulakiBBCCC18},
``blockchain domain-specific languages are difficult to design for the implementer
and require additional learning by the programmer. Writing smart contracts
in a general-purpose language (\emph{e.g.}, Go, Java, C/C++) instead
appears more attractive and accelerates the adoption of blockchain solutions''.
In particular, Java is a well-known programming language,
with modern features such as generics, inner classes, lambda
expressions and lazy streams. For instance, a Solidity contract
for a Ponzi pyramid scheme consists of~39 non-blank non-comment lines of code
(page~155 of~\cite{IyerD18}), while its translation into Java is only 19 lines
long. % (it can be found in the companion archive,
%see Sec.~\ref{sec:experiments}).
Java has a large and powerful toolbelt and an active community.
Also other blockchains~\cite{aion,aion_example_contract,neo,neo_contract,Spoto19}
let programmers code in a limited subset of Java.
It is true that Solidity has native features for smart contracts, such as
\<payable> functions, but the same can be obtained in Java through instrumentation~\cite{Spoto19}.

One of the most compelling reasons for writing smart contracts
in a mainstream language such as Java is that it comes with a large
support library, that provides general solutions to typical programming problems.
Many programmers are familiar with that library and appreciate the possibility
of using it also for developing smart contracts. This reduces
development time and errors, since the library has been widely tested
in the last decades and its semantics is well-known. There is, however, a big
issue here. Namely, the support library of Java is non-deterministic, in general.
For most blockchains, non-determinism leads to a fork of the network, since
consensus cannot be reached. Hyperledger Fabric allows instead non-determinism,
in the sense that code execution, if non-deterministic, gets rejected~\cite{Vukolic17}.
Hence, also in this case, programmers should avoid non-determinism, if they want their code to be run.
Non-determinism can be allowed in smart contracts for the generation of random numbers~\cite{ChatterjeeGP19}, but that
technique does not apply to other forms of non-determinism.

Non-determinism is obvious for library classes and methods that perform, explicitly,
parallel or random computations. But the real problem is that
also some library parts, that are explicitly sequential, might lead to
non-deterministic results.
This is well-known to Java programmers and has been at the origin of subtle
bugs also in traditional software. For smart contracts, however,
this situation cannot be tolerated: the execution of the same code, from the same state,
\emph{must} lead to the same result in any two distinct blockchain nodes.

A solution could be to write a special Java library, whose
methods are made deterministic. But this is far from simple, since the Java library is huge
and determinism would require to modify very low-level aspects of the same virtual machine, such as
its memory allocation strategy and its garbage collectors.
The process should be repeated at each new version of Java.
Moreover, programmers should be aware to use a non-standard library, with a different semantics.
Finally, one should be sure to have fixed \emph{all} possible sources of non-determinism in the
(immense) Java library.
Instead, this paper advocates the use of a standard, fixed Java library, restricted to a white-listed
fragment and with the addition of a verification layer that enforces some run-time conditions.
It must be ensured that this fragment is deterministic, which is typically small (but still
large if compared to, say, Solidity, that has \emph{no} native support library and only very
rudimentary third-party libraries; in particular, Solidity's \<library> keyword
only allows one to define stateless libraries). One needn't prove
to have caught every single potential case of non-determinism in the library, but only that
the white-listed sandbox is safe.

Namely, the contributions of this paper are a discussion of the (non-)determi\-nism
of the Java library methods; a way for specifying deterministic fragments
of that library; a technique for
enforcing some run-time conditions needed for determinism, statically or dynamically;
an implementation of the technique, that is currently part of the verification
layer for Java smart contracts of the Takamaka blockchain;
experiments with that implementation.
Its deterministic fragment has been specified manually, since there is
currently no automatic way of proving that a method is deterministic.
The Takamaka blockchain verifies and then instruments the code of the smart contracts, that gets
then installed, in instrumented form, in blockchain~\cite{Spoto19}. Later, transactions execute the
instrumented code, never the original one. All nodes perform verification and instrumentation
and must agree on the result. An attacker cannot modify the instrumented program
or its execution semantics after it has been installed in blockchain.
The motivation of the original instrumentation goes beyond the issue on determinism;
we refer to~\cite{Spoto19} for its description, while this paper describes its use for enforcing
determinism only.

This work is organized as follows.
Sec.~\ref{sec:determinism} discusses the (non-)determinism of some
examples of Java library methods.
Sec.~\ref{sec:white_listing} defines the notion of \emph{deterministic white-listed fragment} of the Java library.
%that is, a deterministic subset of the library that can be used for writing smart contracts.
Its definition can use run-time conditions on the way its methods are called.
Sec.~\ref{sec:enforcing} provides a technique for enforcing such conditions, dynamically
or statically, through bytecode instrumentation.
Sec.~\ref{sec:experiments} reports experiments with an implementation.

This paper is \emph{not} about Java smart contracts themselves. The interested reader
is referred to the references provided above and, in particular, to~\cite{Spoto19}
and to the tutorial about Takamaka smart contracts in
Java\footnote{Available at \url{https://www.takamaka.dev/docs/Takamaka.html}}.
%Hence, this paper does not tackle
%the specific syntax, methods and technology needed for writing smart contracts in Java.
%Examples are presented in an abstract way, unrelated to any specific
%Java dialect for smart contracts or any specific blockchain.
