\documentclass[orivec]{llncs}
\usepackage{amsmath}
\usepackage{amssymb}
\usepackage{wrapfig}
\usepackage{stmaryrd}
\usepackage{graphicx}
\usepackage[T1]{fontenc}
\usepackage{array}
\usepackage{listings, framed}
\lstset{
  language=Java,
  showstringspaces=false,
  columns=flexible,
  basicstyle={\ttfamily},
  frame=l,
  numbers=left,
  numberstyle={\ttfamily},
%  breaklines=true,
  breakatwhitespace=true,
  tabsize=3,
  escapechar=|
}
\usepackage{algorithm, algorithmic}
\usepackage{url}
\usepackage{relsize}
\usepackage{xcolor}
\usepackage{multirow}

\newcommand{\PreserveBackslash}[1]{\let\temp=\\#1\let\\=\temp}
\newcolumntype{C}[1]{>{\PreserveBackslash\centering}p{#1}}
\newcolumntype{R}[1]{>{\PreserveBackslash\raggedleft}p{#1}}
\newcolumntype{L}[1]{>{\PreserveBackslash\raggedright}p{#1}}

\newif\iflongversion
\longversionfalse               % for conference version
%\longversiontrue                % for technical report

% \|name| or \mathid{name} denotes identifiers and slots in formulas
\def\|#1|{\mathid{#1}}
\newcommand{\mathid}[1]{\ensuremath{\mathit{#1}}}
% \<name> or \codeid{name} denotes computer code identifiers
\def\codesize{\smaller}
\def\<#1>{\codeid{#1}}
\newcommand{\codeid}[1]{\ifmmode{\mbox{\codesize\ttfamily{#1}}}\else{\codesize\ttfamily #1}\fi}

\newcommand{\todo}[1]{{\color{red}\bfseries [[#1]]}}

\newcommand{\fs}[1]{\todo{FS: #1}}

% Reduce padding for \boxed
\setlength{\fboxsep}{.5\fboxsep}

\newcommand{\wrt}{\textit{w.r.t.\ }}
\newcommand{\ie}{, \textit{i.e.}, }
\newcommand{\loc}{\mathbb{L}}
\newcommand{\nil}{{\ensuremath{\mathord{\codeid{null}}} }}
\newcommand{\sep}{\mathop{\mid\!\mid}}
\newcommand{\append}{\!::\!}
\newcommand{\locations}{\ensuremath{\mathbb{L}}}
\newcommand{\objects}{\ensuremath{\mathbb{O}}}
\newcommand{\classes}{\ensuremath{\mathbb{K}}}
\newcommand*{\tto}{{\begin{array}{c}\rightarrow\\[-1ex]\rightarrow\end{array}}}
\newcommand*{\ttob}{{\tto
    \mbox{\scriptsize$\begin{array}{c}b_1\\[-1ex]\cdots\\[-1ex]b_m\end{array}$}}}
\newcommand*{\ttotwo}[2]{{\tto
    \mbox{\scriptsize$\begin{array}{c}{#1}\\[-1ex]{#2}\end{array}$}}}
\newcommand*{\ttoone}[1]{{\rightarrow
    \mbox{\scriptsize$\begin{array}{c}{#1}\end{array}$}}}
\newcommand{\inputv}[1]{\check{{#1}}}
\newcommand{\outputv}[1]{\hat{{#1}}}
\newcommand{\inputt}{\inputv{\mathit{tainted}}}
\newcommand{\outputt}{\outputv{\mathit{tainted}}}
\newcommand{\inter}[1]{\left[\!\left[{#1}\right]\!\right]}
\newcommand{\interlarge}[1]{\left[\!\!\left[{#1}\right]\!\!\right]}
\newcommand*{\block}[2]{\fbox{\scriptsize${#1}$}^{\,{#2}}\hspace*{-1ex}}
%
\begin{document}
%
\begin{frontmatter}
\title{Enforcing Determinism of Java Smart Contracts}
\author{Fausto Spoto}
\institute{Dipartimento di Informatica, Universit\`a di Verona, Italy\\
  \email{fausto.spoto@univr.it}}
%
\maketitle
%
\begin{abstract}
  Java is a high-level, modern, well-known and powerful object-oriented programming
  language, with a large support library and a comfortable toolbelt.
  Hence, it is progressively being proposed as a programming language for
  smart contracts. However, its support library is non-deterministic,
  which is a blocking issue for the application to smart contracts.
  This paper discusses the kind of (non-)determinism of the methods
  of the Java library and how a deterministic fragment of that library can be specified,
  that can be used for writing smart contracts. Moreover, it
  shows that some relevant parts of the library can be used only under specific
  conditions (\emph{proof-obligations}) on run-time values,
  and concludes with the description of
  an instrumentation that enforces such proof-obligations, statically or dynamically.
  The technique has been implemented for the [removed for blind review] blockchain.
\end{abstract}
%
\end{frontmatter}

\section{Introduction}\label{sec:introduction}

Smart contracts are programs that specify the effects of running blockchain transactions.
They are written in specialized programming languages, that take into account
the fact that smart contracts operate on data kept in blockchain. They must
support some special concepts, such as the access to the caller of a transaction,
monetary transfers between contracts
and payment of code execution through \emph{gas}.
Such concepts are not natively available
in traditional programming languages.
Instead, the Bitcoin bytecode~\cite{Antonopoulos17,Nakamoto08}
can be seen as a language for programming smart contracts,
although non-Turing equivalent and mostly limited to coin transfers. The more
powerful Solidity language~\cite{AntonopoulosW18},
compiled into Ethereum bytecode, allows one to code
complex smart contracts, in an imperative high-level language, and is
one of the main reasons behind the success of the Ethereum blockchain.

Recently, there have been efforts towards the use of traditional
high-level languages for writing smart contracts. In particular, Java has
been chosen since it is a well-known programming language,
with modern features such as generics, inner classes, lambda
expressions and lazy streams. Technically, Java has
a large and powerful toolbelt and an active community.
These efforts~\cite{aion,aion_example_contract,neo,neo_contract}
[extra reference removed for blind review]
typically allow programmers to write code in a limited subset of Java
and of its support library,
with extra methods and classes to implement contracts, refer to their
callers and implement monetary transfers.

One of the most compelling reasons for writing smart contracts
in a mainstream language such as Java is that it comes with a large
support library, that provides general solutions to typical programming problems.
Many programmers are familiar with that library and appreciate the possibility
of using it also for developing smart contracts. This reduces
development time and errors, since the library has been widely tested
in the last decades and its semantics is well-known. There is, however, a big
issue here. Namely, the support library of Java is non-deterministic, in general,
while smart contracts must be deterministic, for reaching consensus.

Non-determinism is obvious for library classes and methods that support, explicitly,
parallel or random computations. But the real problem is that
also some library parts, that are explicitly sequential, might lead to
non-deterministic results.
This is well-known to Java programmers and has been at the origin of subtle
bugs also in traditional software. For smart contracts, however,
this situation cannot be tolerated: the execution of the same code, from the same state,
\emph{must} lead to the same result in any two distinct blockchain nodes.

The contributions of this paper are a discussion of the (non-)determinism
of the Java library methods; a way for specifying deterministic fragments
of that library; and a technique for
enforcing some run-time conditions needed for determinism, statically or dynamically.
We have implemented this technique, that is currently part of the verification
layer for Java smart contracts of the [reference removed for blind review] blockchain.

This work is organized as follows.
Sec.~\ref{sec:determinism} discusses the (non-)determinism featured by some
examples of Java library methods.
Sec.~\ref{sec:white_listing} defines the notion of \emph{deterministic white-listed fragment} of the Java library,
that is, a deterministic subset of the library that can be used for writing smart contracts.
Its definition can use run-time conditions on the way its methods are called.
Sec.~\ref{sec:enforcing} provides a technique for enforcing such conditions, dynamically
or statically, through bytecode instrumentation.

This paper is \emph{not} about Java smart contracts themselves. The interested reader
is referred to the references provided above. Hence, this paper does not tackle
the specific syntax, methods and technology needed for writing smart contracts in Java.
Examples are presented in an abstract way, unrelated to any specific
Java dialect for smart contracts or any specific blockchain.

\section{Determinism for Java Methods in Smart Contracts}\label{sec:determinism}

This section discusses the different kinds of determinism of methods from
the Java library and what must be required if they are used
in a language for smart contracts.
We recall that the Java library exists in different versions, from its first
$1.0$ edition of 1996 to its $13$ edition of September 2019.
Each version has different implementations, with OpenJDK probably being
the most used, nowadays. Some classes and methods exist only in some versions of the library.
For instance, class \<java.lang.Integer> exists from the very first $1.0$ edition,
while class \<java.util.Collection> was only introduced with version $1.2$.
%In the following, when discussing a specific method, we assume that a version
%of the Java library is used, where the method (and hence its declaring class) exists.
%We report fully-qualified names of classes only the first time they are cited.

Let us discuss the meaning of \emph{determinism}.
The \emph{state} $\sigma$ of the Java interpreter is typically defined
as a function from variables and object fields into values.
Then the goal is to guarantee that the execution $\phi$ of a piece of code,
from the same state $\sigma$ but in two different blockchain nodes, yields the same result, \emph{i.e.},
same next state $\sigma'$ and same value $r$ (if the code is an expression):
$\phi(\sigma)=\langle\sigma',r\rangle$ in all nodes of the network.
This definition, however, is too strict. Consider the constructor
\<String()> of \<java.lang.String>, that instantiates and returns a new empty \<String>
object. It is an expression whose execution,
from any given state $\sigma$ but in two different blockchain nodes,
will very likely yield two distinct
pointers $r$ in RAM: the heap allocation will likely
pick two distinct free locations.
But both references will refer to a brand new empty string.
As long as the programming language does not allow one to distinguish
the exact pointer, but only the object it refers to,
then such two next states and results can be considered \emph{equivalent}.
This notion of state and reference equivalence is actually borrowed
from~\cite{BanerjeeN05,BarthePR13}.
With this interpretation in mind, \<String()>, applied to two equivalent
states but in two distinct blockchain nodes,
yields equivalent states and equivalent results (two references to
brand new empty strings), and is consequently deterministic.

\begin{figure}[t]
  \begin{center}
    {\small\begin{tabular}{|l||C{1cm}|C{1.25cm}|C{1.57cm}|C{1cm}|}
      \hline
      \multicolumn{1}{|c||}{\multirow{3}{*}{\textbf{Method}}} & \multicolumn{4}{ c| }{\textbf{Deterministic?}}\\\cline{2-5}
      & \multirow{2}{*}{always} & \multirow{2}{*}{platform} & platform + & \multirow{2}{*}{never} \\
      & & & conditions & \\\hline\hline
      \<Number.intValue()> & \checkmark &   &   & \\\hline
      \<Object()> & \checkmark &   &   & \\\hline
      \<String()> & \checkmark &   &   & \\\hline
      \<HashSet$\text{<}$E$\text{>}$()> & \checkmark &   &   & \\\hline
      \<String(String original)> & \checkmark &   &   & \\\hline
      \<String.concat(String other)> & \checkmark &   &   & \\\hline
      \<String.length()> & \checkmark &   &   & \\\hline
      \<Integer.valueOf(int i)> & & \checkmark & & \\\hline
      \<Object.hashCode()> & & & \checkmark & \\\hline
      \<Object.toString()> & & & \checkmark & \\\hline
      \<Collection$\text{<}$E$\text{>}$.iterator()> & & & \checkmark & \\\hline
      \<Collection$\text{<}$E$\text{>}$.stream()> & & & \checkmark & \\\hline
      \<Collection$\text{<}$E$\text{>}$.add(E e)> & & & \checkmark & \\\hline
      \<StreamSupport.stream(Spliterator$\text{<}$T$\text{>}$ s, boolean p)> & & & \checkmark & \\\hline
      \<System.currentTimeMillis()> & & & & \checkmark \\\hline
      \<BaseStream$\text{<}$T,S$\text{>}$.parallel()> & & & & \checkmark \\\hline
      \<Thread.start()> & & & & \checkmark \\\hline
      \multicolumn{1}{c}{} & \multicolumn{3}{c}{\upbracefill}\\
      \multicolumn{1}{c}{} & \multicolumn{3}{c}{potentially white-listed}\\
    \end{tabular}}
  \end{center}
  \caption{Some methods of the Java library, with their behavior \emph{wrt.} determinism.
\emph{Always} means that a method does not compromise code determinism.
\emph{Platform} means that
a method does not compromise determinism, but only if a specific implementation
of the library is fixed. \emph{Platform + conditions} means that a method
does not compromise determinism, but only if an implementation of the library is fixed
\emph{and} the program satisfies extra conditions, that typically refer to
the actual arguments passed to that method or to other methods at run time.
\emph{Never} means
that a method can have different behaviors in distinct executions,
even on a specific library implementation, and no
condition can be sensibly devised to make its behavior deterministic.}\label{fig:determinism}
\end{figure}

Fig.~\ref{fig:determinism} reports examples of Java library methods, classified
on the basis of the kind of determinism that holds for them.
We discuss them below. We recall that, in Java, a method call
\<o.m(pars)> specifies its \emph{receiver} \<o> and
its \emph{static target} \<C.m(types)>, that is, a signature
reporting the class \<C> from where method \<m> with formal
arguments of type \<types> must be looked for.
Note that \<C> and \<types> do not include generic type parameters, if any, since
they are erased during compilation into Java bytecode and our
instrumentation works on bytecode.
Since Java is an object-oriented language, the static target is just the specification
of the method implementation that must be run (the \emph{dynamic} target):
a non-static method call runs the implementation of \<C.m(types)>
that is selected by looking up \<m(types)> from the run-time class of \<o> upwards,
along the superclass chain\footnote{Java also allows so-called \emph{special} calls, such as
   \<super.m()>, that start the look-up from a given \emph{static}
   type; as well as calls embedded in closures, such as method references
   (corresponding to \<invokedynamic> in Java bytecode). For simplicity,
   these calls are not discussed here, but our implementation deals with them.}.
Hence, any implementation of \<m(types)> in \<C> or in any of its
subtypes can be run. For instance, a call \<o.intValue()> with static target \<Number.intValue()>
can be called on any object \<o> that extends
\<java.lang.Number>. At run time, it might
execute any implementation of \<intValue()> in any subtype of \<Number>,
such as in \<Integer> or in \<java.lang.Double>, depending on the run-time type of \<o>.
Hence, when Fig.~\ref{fig:determinism} classifies \<Number.intValue()> as always
deterministic, this applies to every implementation of
\<intValue()> in the subtypes of \<Number> of the Java library.

%\subsubsection*{Always Deterministic Methods.}
Many methods of the Java library are clearly deterministic. Their behavior is fixed
by the official documentation by Oracle
and does not change across distinct versions of the library. An example is the unwrapping
method \<intValue()> of \<Number> (Fig.~\ref{fig:determinism}). It yields
the primitive \<int> value corresponding to an instance of the abstract class \<Number>, such
as objects of class \<Integer> or \<Double>. For instance,
%\[
\<new Integer(3).intValue() == 3>
%\]
holds in any version of the Java library, as well as
%\[
\<new Double(3.14).intValue() == 3>
%\]
since, in Java, truncation of \<double> into \<int> is machine-independent.
Hence, a language for smart contracts, that must require determinism,
can safely allow the use of that method, always. Other examples
are the constructors of \<String>, \<java.lang.Object> and \<java.util.HashSet>
reported in Fig.~\ref{fig:determinism}, or methods
\<String.concat(String other)>, that yields the concatenation of strings \<this> and \<other>,
and \<String.length()>, that yields the length of a string.
If only these methods are used, then any Java library can be used by any node of a blockchain,
even distint versions in distinct nodes.

%\subsubsection*{Platform-Deterministic Methods.}
The static method \<Integer.valueOf(int i)> wraps
a primitive \<int> value into an object of class \<Integer>. It might be surprising,
but its result can be different in two blockchain nodes, if they
use distinct implementations of the Java library. For instance, while
%\[
\<Integer.valueOf(3) == Integer.valueOf(3)>
%\]
holds in every implementation of the library, since the official documentation requires this method
to cache values between $-128$ and $127$, inclusive, there is instead no guarantee that caching is
used outside that range. Hence
%\[
\<Integer.valueOf(2019) == Integer.valueOf(2019)>
%\]
might be true in some implementations of the Java library
and false in others (such as in Oracle JDK 13.0.1). % that yield distinct objects for the two calls.
We call \emph{platform deterministic} such methods, since they are deterministic only once a specific
implementation (\emph{platform}) of the Java library is fixed.
If only methods of this and of the previous category are used, then
all nodes of a blockchain must run on a given, fixed Java library, to guarantee
determinism. % as part of the consensus rules.

%\subsubsection*{Platform-Deterministic Methods under Specific Conditions.}
Consider method \<Object.hashCode()> of class \<Object> now. It yields an \<int>
hash of its receiver. Its implementation computes that hash from the
RAM pointer value of the object reference.
Since two blockchain nodes will likely use different RAM pointers,
this method is non-deterministic.
In other terms, this method exposes an execution detail (the exact RAM pointer)
that was meant to be invisible for state equivalence.
For instance,
\<int i = new Object().hashCode()>
will likely assign different values to \<i>
in two distinct blockchain nodes
(and even in repeated executions on the same node).
The same problem occurs for \<Object.toString()> that, inside
\<Object>, is implemented in terms of \<Object.hashCode()> (its
implementation concatenates
the name of the run-time class of its receiver with its hash and returns it).
Hence, \<String s = new Object().toString()>,
in two distinct blockchain nodes,
will likely assign different strings to variable \<s>.
However, banning calls to \<Object.hashCode()> or \<Object.toString()>
from smart contracts would be unacceptable to programmers, that
heavily use such calls in their programs, without incurring in
non-determinism. The reason is that programmers normally
take care of calling such methods only on objects that redefine
the default (non-deterministic) implementation of
\<hashCode()> and \<toString()> in class \<Object>.
If that is the case, the calls will actually
execute the deterministic redefinitions. Hence, it seems sensible
to allow calls to such methods in smart contracts, but only if their receiver
redefines them in a deterministic way, as in:
\<String o = ...; int h = o.hashCode()>,
where \<o> holds a \<String>, that redefines \<hashCode()>.
%
%\noindent
%or even in the following:
%
%{\small
%\begin{verbatim}
%String s = ... // deterministic computation
%Object o = s;
%int h = o.hashCode(); // o holds a String, that redefines hashCode()
%\end{verbatim}}
%
%\noindent
%In both cases,
%Variable \<o>, at run time,
%holds a string and the deterministic (but potentially platform-specific)
%redefinition of \<hashCode()> inside \<String> will be called.
Hence, such methods are \emph{platform-deterministic under certain conditions}:
they are deterministic if a given Java library is fixed \emph{and} some run-time conditions hold.
Sec.~\ref{sec:enforcing} shows how such conditions can be enforced.

Consider methods \<iterator()> and \<stream()> of the generic
\<Collection$\text{<}$E$\text{>}$>. They provide two ways for
processing the elements, of type $E$, of a collection. The former implements
the traditional \emph{iterator pattern} and yields an object that enumerates
the elements. The latter yields a \emph{stream}, \emph{i.e.},
a lazy algorithm on the
elements, that can be subsequently programmed and executed.
Streams implement the map/filter/reduce pattern, making
heavy use of lamba expressions.
Interestingly, neither method
guarantees a fixed enumeration order.
There are collections for which they
guarantee an order, such as instances of \<java.util.List$\text{<}$E$\text{>}$>:
on lists, enumeration proceeds from head to tail;
or instances of \<java.util.LinkedHashSet$\text{<}$E$\text{>}$>, on which they proceed in
insertion order.
For collections such as \<java.util.HashSet$\text{<}$E$\text{>}$>, instead, the order
varies with the library version \emph{and} at each execution. The reason is that
\<HashSet> uses a hashmap~\cite{CormenLRS09}
to store elements with the same \<hashCode()> in the same bucket.
Since \<hashCode()>, as shown above, is non-deterministic, then
the distribution of elements in the buckets varies from run to run and their enumeration
as well, being the lexicographic scan of the buckets' elements.
One could forbid \<HashSet> and only allow its
more expensive sibling \<LinkedHashSet>, whose iteration order is fixed.
But then method \<add()> would still be non-deterministic, since its gas consumption
is affected by the shape of the buckets, as discussed later.
Moreover, programmers use \<HashSet> extensively
(it is possibly the fifth most used library class:
\url{https://javapapers.com/core-java/top-10-java-classes})
and would be annoyed if it were
banned. It is much better to observe that,
if the \<hashCode()> of all its elements has been redefined in a deterministic way
and if a specific library version is fixed, then the behavior of \<HashSet> becomes
deterministic, since the shape of the buckets is the same in every run. This means that one
can allow, in smart contracts, calls to \<iterator()> and \<stream()> on any collection,
but only under such conditions. Sec.~\ref{sec:enforcing} shows how they can be enforced.

Consider method \<add(E e)> of \<Collection$\text{<}$E$\text{>}$>.
It adds an element \<e> to the collection. For a given library version,
it is deterministic on lists: it adds \<e> to the end of the list.
However, on \<HashSet> and \<LinkedHashSet> it scans the bucket selected
for \<e.hashCode()> and checks if an equal element was already in that
bucket, by calling \<equals()> against each of its elements. Hence \<add(E e)>
on a hashset consumes an amount of gas that depends on the shape of its buckets.
Again, the solution is to require that all elements of the set and \<e> redefine
\<hashCode()> in a deterministic way, which must be enforced at run time
(Sec.~\ref{sec:enforcing}).

Consider the static method
\<stream(Spliterator$\text{<}$T$\text{>}$ s, boolean p)>
of class \<java.util.stream.StreamSupport>. It
yields a stream for processing the elements specified by
the given \<Spliterator>. If the \<Spliterator> is deterministic,
the resulting stream is deterministic as well,
on a given library version, but only if it is sequential.
Passing true for \<p> would yield a parallel stream instead, that is
inherently non-deterministic.
Hence, this method can be used only if a specific library is used and
if it is enforced that false is passed for \<p> at run time (Sec.~\ref{sec:enforcing}).

If only methods of this and of the previous two categories are used, then
all nodes of a blockchain must be run on a given, fixed Java library, \emph{and}
the run-time conditions that entail determinism must be somehow enforced.

%\subsubsection*{Never Deterministic Methods.}
Static method \<System.currentTimeMillis()>, in \<java.lang.System>,
yields the number of milliseconds elapsed since the beginning of 1970.
Not surprisingly, it will yield different measures in different blockchain nodes.
Such an inherently non-deterministic method cannot be used in a smart contract.
Consider method \<parallel()> of \<java.util.stream.BaseStream> now.
It yields a parallel version of a stream. For instance, if \<list> is a list with
at least two distinct positive \<Integer>s:
%
%List<Integer> list = ...
\begin{verbatim}
int pos = list.stream().parallel().mapToInt(Integer::intValue)
  .filter(i -> i > 0).findAny().getAsInt();
\end{verbatim}
%
%and assume that \<list> contains at least two distinct positive \<Integer>s.
%This code
processes \<list> with a parallel algorithm
that unwraps each \<Integer> element of the list into its corresponding \<int> primitive value,
filters only the positive values and selects any of them.
The result of \<findAny()> is an optional value, hence the
\<getAsInt()> call at the end. Since the algorithm is
parallel, each execution of this code might select a different
positive element, depending on thread scheduling. Hence, \<parallel()>
introduces non-determinism and
cannot be allowed in smart contracts. Note that even the gas consumption
of the code is not deterministic, since it depends on how many elements
are checked before a thread encounters a positive value and terminates
the look-up.

In general, methods that introduce parallelism are never deterministic.
Another example is \<Thread.start()> in \<java.lang.Thread>, that spawns
a parallel execution thread.
If methods of this category are used, then the code cannot be used in blockchain,
since there is not way to make it deterministic.

\section{Specifying White-Listed Fragments of the Java Library}\label{sec:white_listing}

Sec.~\ref{sec:determinism} has shown that some methods of the Java library can be used
in smart contracts, at least if a specific version of the library is fixed and some
run-time conditions are enforced. Such methods are called \emph{white-listed}
(Fig.~\ref{fig:determinism}). A set of white-listed methods must be part of the
consensus rules: each node verifies that the methods used in smart contracts are
in the white-listed set and otherwise rejects the installation of the smart contract
in blockchain. This section provides a compact way of specifying the set of white-listed methods,
in a form that can be compiled and provided to the blockchain nodes.

\section{Proof Obligations for Determinism}\label{sec:proof_obligations}

\section{Enforcing Run-Time Conditions for Determinism}\label{sec:enforcing}

Sec.~\ref{sec:white_listing} shows that a deterministic fragment of the Java library
can require run-time conditions on the values of the receiver or
parameters of its methods. A blockchain node must enforce that such conditions hold
at run time for the smart contracts that it executes. Hence each condition
is a proof-obligation that must be discharged: if this is not possible, the smart contract
cannot be executed or, at least, its execution must be aborted. This section shows how
this is possible.

Very likely, a blockchain node receives the smart contract already compiled in Java bytecode.
The idea is hence to let the node instrument such bytecode, only the first time it is installed in blockchain,
with extra checks that, at each subsequent execution, verify the run-time conditions.
To make the technique more accessible to the reader, this section presents
the instrumentation at source-code level, but it actually works at bytecode level.

Assume that a blockchain node verifies that smart contracts obey to the deterministic
fragment in Figs.~\ref{fig:white_listed_Object_Collection_Set}
and~\ref{fig:white_listed_List_HashSet_ArrayList}.
Assume that a user installs in blockchain a smart contract whose
code contains \<collection.remove(element)>, whose static target is
\<Collection$\text{<}$E$\text{>}$.remove(Object o)>.
The node spots this syntactically.\footnote{The Java bytecode of the
  smart contract will
  contain an instruction \<invokeinterface java.util.Collection.remove(Object):boolean>,
  or a similar one for a subtype of \<Collection>.}
The node consults its white-listed fragment and recognizes the call as
white-listed, but having a run-time constraint \<@MustRedefineHashCode> on \<o>
(Fig.~\ref{fig:white_listed_Object_Collection_Set}). Hence, during installation of the
smart contract, the node instruments its code by adding a brand new method:

{\scriptsize\begin{verbatim}
private static boolean verifier_0(Collection<E> receiver, Object par_0) {
  Support.mustRedefineHashCode(par_0);
  return receiver.remove(par_0);
}
\end{verbatim}}

\noindent
and replaces \<collection.remove(element)> with \<verifier\_0(collection, element)>.
Each time that code will later be executed,
\<Support.mustRedefineHashCode(par\_0)> will check (through Java reflection)
that the actual argument
passed to \<remove(Object o)> redefines \<hashCode()> and aborts
the current transaction otherwise. The node includes a
\<Support> class for that, whose
code is not reported for space limitations.

%{\small\begin{verbatim}
%public class Support {
%  public static void mustRedefineHashCode(Object value) {
%    if (value != null)
%      if (Stream.of(value.getClass().getMethods())
%          .filter(method -> !Modifier.isAbstract(method.getModifiers())
%                         && Modifier.isPublic(method.getModifiers())
%                         && method.getDeclaringClass() != Object.class)
%          .map(Method::getName)
%          .noneMatch("hashCode"::equals)) {
%        // abort the current transaction and undo all its side-effects
%      }
%  }
%}
%\end{verbatim}}

%\noindent
%The code above\footnote{This is code in the blockchain node, not in a smart contract:
%  any method can be used here; there is no white-listing notion for it.}
%checks if the value is non-\<null> and has no method
%named \<hashCode>
%(possibly inherited) that is public, non-abstract and not declared in \<Object>.
%If that is the case, then the value does not redefine \<hashCode()> from \<Object>
%and the transaction gets aborted.

For another example, assume that the smart contract calls
\<x.toString()>, with static target \<Object.toString()>.
The blockchain node spots this syntactically,
consults its white-listed fragment and recognizes the call as
white-listed, but having a run-time constraint \<@MustRedefineHashCodeOrToString> on
\<x> (Fig.~\ref{fig:white_listed_Object_Collection_Set}).
Consequently, it instruments the code of the smart contract by adding:

{\scriptsize\begin{verbatim}
private static verifier_1(Object receiver) {
  Support.mustRedefineHashCodeOrToString(receiver);
  return receiver.toString();
}
\end{verbatim}}

\noindent
The node replaces \<x.toString()> with \<verifier\_1(x)>.
Later, at each run of the contract,
\<Support.mustRedefineHashCodeOrToString(receiver)> will be executed,
that checks the condition by reflection.

Assume that the smart contract contains a static call
\<StreamSupport.stream(s, p)>, whose static target is
\<StreamSupport.stream(Spliterator$\text{<}$T$\text{>}$ s, boolean p)>,
and that the white-listed fragment of the blockchain node allows that signature,
but has a run-time condition on \<p> that avoids the creation of parallel streams:
\<StreamSupport.stream(Spliterator$\text{<}$T$\text{>}$ s, @MustBeFalse boolean p)>.
The node instruments the smart contract by adding:

{\scriptsize\begin{verbatim}
private static Stream<T> verifier_2(Spliterator<T> par_0, boolean par_1) {
  Support.mustBeFalse(par_1);
  return StreamSupport.stream(par_0, par_1);
}
\end{verbatim}}

\noindent
Then it replaces \<StreamSupport.stream(s, p)> with \<verifier\_2(s, p)>.
Method \<Support.mustBeFalse(value)> is defined in class \<Support> and
will abort future transactions if \<value> holds false.

%{\small\begin{verbatim}
%public static void mustBeFalse(boolean value) {
%  if (value) {
%    // abort the current transaction and undo all its side-effects
%  }
%}
%\end{verbatim}}

%The name of the verification methods (\<verifier\_0>, \<verifier\_1> and so on)
%must be chosen in such a way to avoid name clashes with already existing methods.
%Moreover, if a call is repeated
%in the smart contract, it is more efficient to share the same verifier for all
%its occurrences and keep the class file smaller.

\subsection{Static vs.\ Dynamic}\label{subsec:static_vs_dynamic}

The instrumentation technique described above adds dynamic checks on run-time values,
that will be triggered during each subsequent transaction.
Hence, checks are performed repeatedly, every time an annotated white-listed method
is executed. This can incur in a performance penalty. It would be better
to check, once and for all,
if a run-time condition holds, definitely, when smart contracts
are installed in blockchain.
This can be done with static analysis~\cite{NielsonNH99}, a technique that infers
properties of programs, before they are actually run. Since the verification of
non-trivial run-time program properties is in general undecidable~\cite{Rice53},
static analysis provides a definite answer
only in some cases. Hence, a blockchain node can use static analysis to discharge
the proof-obligations due to run-time conditions on white-listed methods. If it
succeeds with a definite answer, stating that a given condition definitely holds,
the blockchain node needn't generate any \<verifier> method for that condition. Otherwise, it
generates the \<verifier> method, as last resort.

Static analysis can be more or less aggressive.
More aggressive static analyses discharge more proof-obligations statically,
which is desirable since the smart contract's code will check less
conditions at run-time. However, aggressive analyses are typically more expensive
(although they are executed only once, when the smart contract is installed in
blockchain). In practice, a good trade-off should be found between the power of the analysis
and its cost.

Our implementation uses static type information to infer, statically, if a condition
\<@MustRedefineHashCode> or \<@MustRedefineHashCodeOrToString> holds for a variable \<v>.
If the static type
$\tau$ of \<v> is a class that redefines \<Object.hashCode()> or \<Object.toString()>,
the same must hold for the dynamic type $\tau'$ of \<v>, that can only be an
instance of $\tau$. Hence the run-time condition must hold, always.
This follows from the fact that Java and Java bytecode
are strongly-typed. Otherwise, the \<verifier> method is added for that condition.
For \<@MustBeFalse>, our implementation looks, intra-procedurally, for
the producers of the annotated value.
If these are always the literal \<false>, then the condition holds.
If, instead, at least one producer is the literal \<true> or a complex expression,
then the static analysis gives up and the \<verifier> method is added.

\section{Conclusion}\label{sec:conclusion}

The technique in this paper allows a simple
specification of a deterministic fragment of the Java library and
enforces its run-time constraints on values.
The technique is general, although we focused on the
relevant issue of avoiding \<Object.hashCode()>'s non-determinism.
For that special case, it could be possible to patch the Java library
with an implementation of \<Object.hashCode()> that constantly and
deterministically yields a constant. That would work for this run-time condition
but would be invasive (all blockchain nodes should use a patched,
unofficial version of the Java library). It would also lead to very inefficient
code if objects are put in hashsets without redefining their \<hashCode()>.
Our solution rejects the smart contracts in that case, statically or dynamically.

Our specification and verification technique does not help with the
identification of deterministic fragments of the Java library, that remains
an open problem. Currently,
we have checked the source code of the library to convince ourselves that,
for instance, the fragment in Figs.~\ref{fig:white_listed_Object}, \ref{fig:white_listed_Collection},
\ref{fig:white_listed_Set}, \ref{fig:white_listed_List},
\ref{fig:white_listed_HashSet} and~\ref{fig:white_listed_ArrayList} is deterministic.
Static analysis could help here, if it could prove some fragment deterministic,
automatically. But the complexity of the library code, the need to infer
sensible run-time conditions and the use of native code seem to make this task very hard.


%%%%%%%%%%%%%%%%%%%%%%%%%%%%%%%%%%%%%%%%%%%%
\bibliographystyle{plain}
\bibliography{biblio}

\end{document}
