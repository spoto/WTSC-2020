\section{White-Listed Fragments of the Java Library}\label{sec:white_listing}

\newcommand{\wl}{\mathit{WL}}

Sec.~\ref{sec:determinism} has shown that some Java library methods can be used
in smart contracts, at least if a specific version of the library is fixed and some
run-time conditions are enforced. Such methods can hence be \emph{white-listed} for smart contracts
(Fig.~\ref{fig:determinism}).

\begin{definition}\label{def:white-listing}
  A \emph{white-listed fragment} $\wl$ is a set of method signatures (constructors are considered
  methods named as their defining class), with associated run-time conditions (if any).
\end{definition}
%
The consensus rules must specify a white-listed fragment $\wl$:
each node verifies that smart contracts \emph{obey} $\wl$,
or otherwise aborts their execution in blockchain.
%
\begin{definition}\label{def:obeys}
  A program $P$ \emph{obeys} a white-listed fragment $\wl$ if and only if
  \begin{enumerate}
    \item every method call
      in $P$ has a static target that is either in contract code, or
      in $\wl$ or overrides a signature in $\wl$
      (syntactical check); and
    \item in every execution of $P$, $\wl$'s run-time conditions hold (semantical check).
  \end{enumerate}
\end{definition}
%
A program $P$ that obeys a white-listed fragment $\wl$ cannot call signatures outside
$\wl$, but can call signatures in $\wl$ that, indirectly, call methods outside $\wl$.
In other terms, Def.~\ref{def:obeys} constrains only the library API allowed in $P$.
%
\begin{definition}\label{def:deterministic}
  Given a Java library version $V$, a white-listed fragment $\wl$ is
  \emph{deterministic for $V$}
  if and only if any Java program that obeys $\wl$
  is deterministic, when executed over $V$.
\end{definition}

A very simple example of white-listed fragment is
$\wl_1=\emptyset$. It is deterministic for every Java library version.
Namely, a program $P$ that obeys $\wl_1$ cannot call any library method nor constructor
(condition~1 of Def.~\ref{def:obeys}). $P$ cannot contain classes, since Java classes
have always at least a constructor that calls the constructor of \<Object>, possibly
indirectly. Hence, $P$ consists of interfaces only,
with default and static methods that do not call any library code, and is hence
deterministic.
A deterministic white-listed fragment is $\wl_2=\{\<Object()>\}$.
%It contains
%the constructor of class \<Object> only.
This time, one can write programs $P$
that obey $\wl_2$, with classes that extend \<Object> and whose constructors call \<Object()>.
But no other library methods nor contructors can be called, which is an irrealistic constraint.
The white-listed fragment $\wl_3=\{\<Object()>,\<System.currentTimeMillis()>\}$ is not deterministic,
since it is possible to write a program, that obeys $\wl_3$, consisting of a single class
with a method that uses \<System.currentTimeMillis()> to return a random value.
A deterministic white-listed fragment that allows, at least, simple string manipulations is
$\wl_4=\{\<Object()>,\<String.concat(String other)>,\<String.length()>\}$. It allows one
to write Java programs whose classes extend \<Object> and whose code performs computations
such as \<"hello".concat(s).length()>, where \<s> is a string.

Up to now, we have not used run-time conditions in white-listed fragments.
In order to specify such conditions, it is possible to use Java annotations on
method signatures. An annotation can be applied to a formal parameter, meaning that
the condition must hold for the corresponding actual parameter; or to a non-static method
itself, meaning that the condition must hold for the receiver of the method.
Let us introduce for instance the following annotations:
%
\begin{description}
\item[\<@MustRedefineHashCode>:] the annotated value is \<null> or belongs to a class that redefines
  \<Object.hashCode()>;
\item[\<@MustRedefineHashCodeOrToString>:] the annotated value is \<null> or belongs to a class that redefines
  \<Object.hashCode()> or \<Object.toString()> (or both);
\item[\<@MustBeFalse>:] the annotated value is \<false>.
\end{description}
%
Then one can define
%
\[
\wl_5=\left\{\begin{array}{c}
\mbox{\<Object()>}, \mbox{\<HashSet$\text{<}$E$\text{>}$()>},\\
\mbox{\<@MustRedefineHashCodeOrToString Object.toString()>},\\
\mbox{\<Collection$\text{<}$E$\text{>}$.add(@MustRedefineHashCode E e)>}
\end{array}\right\}
\]
%
that constrains \<Object.toString()> to be called on values that redefine
at least one of \<hashCode()> and \<toString()>, and
\<Collection$\text{<}$E$\text{>}$.add(E e)> to be called
with an actual parameter that redefines \<hashCode()>. $\wl_5$ is deterministic
for any given Java library version, thanks
to such constraints. For instance, the following code is deterministic
on any Java library:

{\small\begin{verbatim}
Set<Object> set = new HashSet<>();
set.add("hello"); set.add(BigInteger.ONE); set.add(new HashSet<String>());
String s = set.toString();
\end{verbatim}}

\noindent
It creates a hashset and populates it with a
\<String>, a \<java.math.BigInteger> and an empty \<HashSet>. These
redefine \<hashCode()>, hence the constraint on \<add()> holds.
At the end, it calls \<toString()> on the
\<set>, that is a \<HashSet> that redefines \<toString()>,
hence the constraint
on \<toString()> holds. By using, for instance, the
OpenJDK~13 library, variable \<s> will always hold the string
\<"[[], 1, hello]">. Other versions of the library might compute different strings but,
once a library version is fixed, always the same string is computed.
The reason of this determinism is that the shape of the buckets of the hashset is fixed
since the \<hashCode()> of its elements is redefined in a deterministic way in
\<String>, \<BigInteger> and \<HashSet>. Moreover, \<toString> on \<HashSet> has been
redefined in a way that iterates on the elements of the hashset and concatenates
their \<toString()>, which is redefined and deterministic in
\<String>, \<BigInteger> and \<HashSet>.

The justification above for the determinism of the code follows from
a manual investigation of the library's source code. It is not automated. What can
be automated is, instead, the verification that a program $P$ obeys a given $\wl$.
Namely, condition~1 of Def.~\ref{def:obeys} can be verified
%in time $O(|P|\cdot h\cdot\log_2|\wl|)$, where $|P|$ is the size of $P$,
%$h$ is the maximal height of the class
%hierarchy and $|\wl|$ is the size of $\wl$.
%Namely,
by following the static target of each call in $P$, upwards,
and checking if the method is in $\wl$.
%If a total
%order is defined on method signatures, this last check can be done through binary search,
%with logarithmic complexity.
Condition~2 of Def.~\ref{def:obeys} is more complex and
Sec.~\ref{sec:enforcing} shows how it can be enforced.

A white-listed fragment $\wl$ must be specified in a way that is easily machine-readable.
$\wl$ is a set of method signatures, hence it can be provided
as a set of Java abstract classes and interfaces. For each library class \<C> that defines some
methods to white-list, one writes an abstract class \<whitelisted.C>;
for each interface \<I>, one write an interface \<whitelisted.I>. That is, one defines,
in package \<whitelisted>, \emph{mirrors} declaring the white-listed signatures and
their annotations, if any.
The advantage of using such mirrors is that they can be written by copying and pasting
signatures from the source code of the mirrored library classes. Moreover, they can be compiled,
which helps spotting typos. Finally, a blockchain node can query such classes by reflection,
which is both simple and efficient, compared for instance to querying textual or XML
specifications.

\begin{figure}
{\scriptsize\begin{verbatim}
public abstract class whitelisted.java.lang.Object {
  public Object() {}
  public abstract boolean equals(java.lang.Object other);
  public abstract @MustRedefineHashCodeOrToString java.lang.String toString();
  public abstract @MustRedefineHashCode int hashCode();
}

public interface whitelisted.java.util.Collection<E> {
  int size();
  boolean isEmpty();
  java.lang.Object[] toArray();
  <T> T[] toArray(T[] a);
  <T> T[] toArray(java.util.function.IntFunction<T[]> generator);
  boolean add(@MustRedefineHashCode E e);
  boolean contains(@MustRedefineHashCode java.lang.Object o);
  boolean remove(@MustRedefineHashCode java.lang.Object o);
  boolean containsAll(java.util.Collection<?> c);
  boolean addAll(java.util.Collection<? extends E> c);
  boolean removeAll(java.util.Collection<?> c);
  boolean removeIf(java.util.function.Predicate<? super E> filter);
  boolean retainAll(java.util.Collection<?> c);
  void clear();
  java.util.stream.Stream<E> stream();
  java.lang.Iterator<E> iterator();
}

public interface whitelisted.java.util.Set<E> {
  boolean containsAll(java.util.Collection<?> c);
  boolean addAll(java.util.Collection<? extends E> c);
  boolean retainAll(java.util.Collection<?> c);
  boolean removeAll(java.util.Collection<?> c);
}

public interface whitelisted.java.util.List<E> {
  E get(int index);
  E remove(int index);
  boolean remove(java.lang.Object o);
  boolean contains(java.lang.Object o);
  void sort(java.util.Comparator<? super E> c);
  E set(int index, @MustRedefineHashCode E element);
  void add(int index, @MustRedefineHashCode E element);
  int indexOf(java.lang.Object o);
  int lastIndexOf(java.lang.Object o);
  java.util.ListIterator<E> listIterator();
  java.util.ListIterator<E> listIterator(int index);
  java.util.List<E> subList(int fromIndex, int toIndex);
  java.util.Spliterator<E> spliterator();
}

public abstract class whitelisted.java.util.HashSet<E> {
  public HashSet() {}
  public HashSet(java.util.Collection<? extends E> c) {}
  public HashSet(int initialCapacity, float loadFactor) {}
  public HashSet(int initialCapacity) {}
}

public abstract class whitelisted.java.util.ArrayList<E> {
  public ArrayList() {}
  public ArrayList(int size) {}
  public ArrayList(java.util.Collection<? extends E> c) {}
  public abstract void trimToSize();
  public abstract void ensureCapacity(int minCapacity);
}
\end{verbatim}}
  \caption{White-listed methods of \<Object>,
    \<Collection>, \<Set>, \<List>, \<HashSet> and \<ArrayList>.}
\label{fig:fragment}
\end{figure}

For instance, Fig.~\ref{fig:fragment}
specifies a white-listed fragment of \<Object>,
\<Collection>, \<Set>, \<List>, \<HashSet> and
\<ArrayList>\footnote{These signatures are copy and paste from the
library source code. For the use of generics, wildcards and
\<Object> in these signatures, we refer to~\cite{NaftalinW06}.}.
It is deterministic for a given fixed version of the Java library:
thanks to its annotations, this fragment guarantees that elements in collections
(hence also in sets and lists) redefine \<hashCode()>, which makes iterators
on collections, streams derived from collections and \<toString()> on collections
deterministic. The latter can be called on collections since they redefine it,
hence satisfying the constraint for \<Object>.

Interestingly, method
\<contains(Object o)> in \<Collection> (Fig.~\ref{fig:fragment})
requires \<o> to redefine \<hashCode()>
or otherwise, for hashsets, the bucket where \<o> is looked for might be different
for different runs, with subsequent non-determinism.
That constraint is not needed for \<contains(Object o)> in \<List>
(Fig.~\ref{fig:fragment}), that scans the list from its head,
looking for an \<equals()> element, deterministically,
regardless of \<o> having redefined \<hashCode()> or not. This is consistent with
Liskov's substitution principle~\cite{LiskovW94} that,
for formal arguments, works by generalization:
hence white-listing constraints on formal arguments can be weakened, not strengthened,
in subclasses. The same for \<remove(Object o)>, passing from \<Collection> to \<List>.

There is no largest deterministic fragment
for a given set of classes and interfaces:
another deterministic fragment is identical to
Fig.~\ref{fig:fragment} but for allowing any elements in lists, possibly not redefining
\<hashCode()> (hence more permissive);
forbidding \<Object.toString()> and \<Object.hashCode()> altogether,
since it would now be non-deterministic on lists (hence
more restrictive); and forbidding the constructor
\<HashSet(Collection$\text{<}$? extends E$\text{>}$ c)>
(Fig.~\ref{fig:fragment}), that allows a hashset to be
built from a list and hence contain elements whose \<hashCode()> has not been redefined,
with consequent non-determinism (hence more restrictive). Choosing a specific
deterministic fragment is often a question of personal taste.
One should choose that allowing more methods of frequent use. In this example,
forbidding \<Object.toString()> and \<Object.hashCode()> would hardly be acceptable.
%In principle, blockchain nodes could know about more deterministic fragments and check
%if a smart contract obeys at least one of them, hence allowing more code.
%In practice, the complexity of dealing with more fragments might not justify the effort.
